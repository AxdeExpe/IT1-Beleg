\documentclass[12pt, letterpaper]{article}



\begin{document}
	\begin{titlepage}
		\title{Beleg Internettechnologien 1 \\ (PWA zum lernen von Modulen) }
		\author{Autor: Kevin Pietzsch, s83799}
		\date{12.04.2023} %/nochmal anpassen
		\maketitle
	\end{titlepage}
	
	\newpage
	\begin{center}
		\title{\Huge{Gliederung}}\\
	\end{center}
	
	\renewcommand\contentsname{Inhalt}
	\tableofcontents
	
	\newpage
	\section{Einleitung}
	In diesem Beleg sollte eine PWA (Progressive Web App), zum lernen von Modulen, erstellt werden.\\
	Dabei musste der Autor HTML5, CSS3 und Javascript nutzen.
	\newpage
	
	
	
	\section{Planung}

	\newpage	
	
	
	
	\section{Entwicklung}

	\newpage
	
	\section{Sonstiges}



	
\end{document}